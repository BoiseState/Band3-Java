\documentclass[project]{bsu-ms}

\usepackage[dvips]{graphicx}

\title{Conversion of the Band Diagram Program:\titleBreak A Look at Portability, Efficiency and Ease of Use.}  

% Author's name (must be exactly what name the Registrar has)
% normally it is of the form "First Middle Last"
\author{Michael Lavern Baker}  

% Day of oral defense
\graduationDay{08}

% Month of graduation (do not abbreviate)
\graduationMonth{October}

% Year of graduation
\graduationYear{2010} 

\advisor{Amit Jain, Ph.D.} % also the chair of your committee
\committeeA{William Knowlton, Ph.D.} 
\committeeB{Teresa Cole, Ph.D.}

% Full name of the degree; normally "Master of Science"
\degree{Master of Science}

% Full name of major
\major{Computer Science}

% Major department
\department{Computer Science}

% College 
\college{Engineering}

% Current department chair (at the moment, this is not used)
\departmentChair{Murali Medidi}

% Current graduate dean
\graduateDean{John R. Pelton}

% Document abstract
\abstract{The Band Diagram program, originally written by Richard Southwick III under the direction of William Knowlton, is an interactive simulation project that is useful for visualizing approximations of energy band diagrams and other related diagrams. The purpose of this project is to convert the existing Band Diagram program, written in C\#, to the more portable Java language. The conversion allows the new version to be used on a wider variety of platforms and aids in furture maintenance. In addition to the conversion, the project also involved implementing enhanced charting functionality, normalizing the interface layout using industry standards and an attempt to improve the processing times of various internal algorithms.}

\includeCopyright

\maxPage{99}

\acknowledgments{
I would like to first and foremost express gratitude to my wife for her support and understanding during this project. Without her this would not have been possible. 

I would like to thank my children for their understanding during the time it has taken to complete this project. 

I would also like to thank Amit Jain, Bill Knowlton and Teresa Cole for their invaluable work as advisors  and committee members on this project.

Lastly, I would like to thank Richard Southwick III for his work on the original Band Diagram program and his help and insight during the conversion process.
}

\begin{document}
\frontmatter
\buildFrontPages

% The list of abbreviations
\begin{listAbbreviations}
\item[CUDA] Compute Unified Device Architecture
\item[GPL] GNU General Public License
\item[GPU] Graphics Processing Unit
\item[GUI] Graphical User Interface
\item[HTML] HyperText Markup Language
\item[IDE] Integrated Development Environment
\item[JNLP] Java Network Launching Protocol
\item[JPEG] Joint Photographic Experts Group
\item[JRE] Java Runtime Environment
\item[JWS] Java Web Start
\item[OS] Operating System
\item[PNG] Portable Network Graphics
\item[XML] eXtensible Markup Language
\end{listAbbreviations}

\mainmatter

%%%%%%%%%%%%%%%%%%%%%%%%%%%%%%%%%%%%%%%%%%%%%%%%%%%%%%%%%%%%%%%%%%%%%%%%%%%%%%
%
% Chapter: Introduction
%
%%%%%%%%%%%%%%%%%%%%%%%%%%%%%%%%%%%%%%%%%%%%%%%%%%%%%%%%%%%%%%%%%%%%%%%%%%%%%%
\chapter{Introduction}\label{ch:intro}

The Band Diagram program is an ``interactive simulation project that is useful for visualizing approximations of energy band diagrams." \cite{band:web1} The Band Diagram program was originally written by Richard Southwick III, under the direction of William Knowlton, in the C\# programming language. After the initial creation of the Band Diagram program and then the release of the enhanced second version, the software has gained popularity among industry and research professionals. 

With this popularity came the desire from users to be able to run the software on Linux and Macintosh operating systems, as well as Microsoft Windows. An initial attempt was made to use the Mono project, ``an open source, cross-platform implementation of C\# and the Common Language Runtime that is binary compatible with Microsoft.NET" \cite{band:web2}, to achieve this goal. However, execution was very unpredictable, especially under the Macintosh operating system. Frequent crashes and slow development (and thus limited functionality), of the Mono project have eliminated this solution. This project involves finding and implementing a cross-platform solution for the Band Diagram Program.



\section{Background}\label{sec:background}

The original Band Diagram program, started in 2005, is ``useful for visualizing approximations of energy band diagrams and performing back of the envelope calculations of important parameters for these diagrams." \cite{band:web1} This initial version of the software utilizes ``basic analytical equations to visualize the energy band diagrams of various dual-oxide MOS structures." \cite{band:paper1} 

\begin{figure}[ht]
\begin{center}
\includegraphics*[width=6.0in,height=5.0in,keepaspectratio]{origBandProgram.eps}
\end{center}
\caption{Original Dual-oxide Band Diagram program}
\label{fig:origProg}
\end{figure}

The main screen of this software can be seen in figure Figure~\ref{fig:origProg}. While limited to only dual-oxide MOS devices, the program supports ``various dielectrics including high K dielectrics. It is capable of calculating parameters such as electric fields, voltage drops, capacitance, EOT, etc. as a function of dielectric thicknesses and applied voltage both in inversion and accumulation." \cite{band:web1}

From 2005 to 2007 the software went through many updates for the following purposes: 

\begin{enumerate}
\item To allow saving image files, 
\item To support exporting data in comma delimited format,
\item To add a variety of dielectric materials, and
\item To improve the algorithms used to increase the accuracy of the software.
\end{enumerate}

While this version was useful, it presented limitations as research progressed. With the limit of only dual-oxide MOS devices, more complex devices could not be explored. Because of these limitations, the newer Multi-Dielectric Energy Band Diagram software was written to resolve these issues\cite{band:paper2}. This new version, written and maintained from 2007 to the present, allowed for an arbitrary number of metals and dielectrics to be used.

It also supports inserting fixed charges into the dielectrics as well as allowing the user to define band gap and intrinsic carrier concentrations as a function of temperature. In addition to these functional changes, more chart views and stack calculations were made available. The adding and editing of stored materials and their constants was also made available to the users of this newer version.

While very functional for the Microsoft Windows family of operating systems, the software was not reliably compatible with other OSes as previously mentioned. Because of the importance of cross-platform capability in the areas of education and research, these incompatibility issues have necessitated this project.



\section{Project Statement}\label{sec:projectStatement}
The purpose of this project is primarily to convert the existing Band Diagram program, to a more portable language. This conversion will facilitate a better experience for users by eliminating crashes and other issues due to the Mono implementation. 

In addition to the conversion, the project also consists of implementing further charting functionality and normalizing the interface layout with industry standard techniques.




%%%%%%%%%%%%%%%%%%%%%%%%%%%%%%%%%%%%%%%%%%%%%%%%%%%%%%%%%%%%%%%%%%%%%%%%%%%%%%
%
% Chapter: Conversion Process
%
%%%%%%%%%%%%%%%%%%%%%%%%%%%%%%%%%%%%%%%%%%%%%%%%%%%%%%%%%%%%%%%%%%%%%%%%%%%%%%
\chapter{Conversion Process}\label{ch:conversionProcess}

This section discusses the conversion process used for the Band Diagram program. This section covers design and implementation decisions and their rational as well as release and testing methods used.

\section{Programming Language}\label{sec:programmingLanguage}
As previously stated, the primary objective of this project is to convert the Band Diagram program from C\# into a more portable language. Through the release of previous versions of the Band Diagram software it was found that many researchers that could potentially utilize the software use Mac OS and cannot benefit from a Microsoft Windows version. Conversion to the Java programming language has many benefits, the strongest of which are related to the language's cross-platform capabilities. The target for the Band Diagram software is for it to run on Microsoft Windows, Linux and Mac OSs. The most efficient way to accomplish this is to use a cross-platform language that natively supports running the same code on all three operating systems. Java provides this functionality and is currently commercially supported yet free software. This support will help to ensure that the cross-platform compatibilities and potentially new OS support can be maintained and added throughout the Java programming language life-cycle.

An alternative to Java would be to use an OS and processor specific language like C++. This would create a need for three distinct code bases and machines with each OS to compile releases. However, it could provide a slight improvement in performance due to the code being optimized for a specific architecture at compile time and having all the compiling completed before execution time. While attempts are made to standardize C++ across different platforms, in reality many of the frameworks and other third party libraries are not available on all platforms. Also, many C++ compilers require specific differences in the code to compile correctly such as access to various OS level functions.

Due to these limitations with C++ and other languages which are compiled to a specific system, these languages were eliminated from consideration. 

Another option would be to use scripting languages such as Perl or Python. These languages have great support for multiple platforms and many freely available libraries that are also cross-platform to enhance the software\cite{band:web4}. Scripting languages, like those mentioned, have serious limitations, however. Scripting languages have some of the worst performance of any of the available programming languages (Table~\ref{tbl:speed}) and often require large runtime libraries to work. 

\begin{table}[h]
\caption{Programming Language Speed Comparison \cite{band:web5}}
\label{tbl:speed}
\centering
\begin{tabular}{|c|c|c|c|}\hline \hline
Language  &25\%  &Median &75\%\\ \hline
C++          &1.01s  &1.10s   &1.29s\\ \hline
Java          &1.59s  &2.34s   &3.90s\\ \hline
Python      &8.09s  &42.19s   &94.54s\\ \hline
Perl           &6.43s  &88.59s   &193.12s\\ \hline
\end{tabular}
\end{table}

An application can also require additional system specific libraries to support the scripting code. Most libraries are stored in code form, which can be edited and viewed by users. Many scripting languages also do not fit well into automatic software installation systems common to desktop applications. It is also common in scripting languages for backward compatibility to take second place to framework redesign and feature enhancements. This can cause frustration when more than one scripting language version is necessary to run programs with different source versions.

For these reasons Java was chosen as the target translation language. Java provides the best of both worlds by providing excellent cross-platform support, yet optimizing speed by using operating system specific compiler improvements for the systems the runtime is available for. 

Another benefit of using a truly cross-platform language like Java is the availability of many third party code libraries which can simplify the conversion process and also allow for easier enhancement of the current feature set.

Finally, Java provides a structured but simple syntax that can be maintained and improved without advanced knowledge of the language mechanics.



\section{Coding Software and Methods}\label{sec:codingSoftwareAndMethods}
There are a few automated methods for converting code from one language to another. Various conversion tools were explored as a possible way to increase the speed of the conversion process. The NetBeans IDE was used to improve general coding speed and simplify form development by providing easy to use access to the Matisse Swing GUI Builder for Java. The JFreeCharts charting package for Java was used as a basis for the charts in the new Band Diagram Program. JFreeCharts provides significant improvement over the hand-coded charts in the current C\# version of the program including additional chart types, multiple series support, auto-scaling, pan and zoom support and support for multiple layout effects\cite{band:web3}.



\section{Automatic C\# to Java Translators}\label{sec:automaticTranslator}
In an attempt to increase the speed of translation, automatic conversion software was reviewed. While there are many projects available today for this purpose, many have a high purchase price. Of the sundry open source solutions available, most are in various and incomplete development stages. It is common to find statements comparing C\# and Java syntax. Both are said to be very structured and similar. 

Why then is a conversion tool so difficult or time consuming to create? The answer lies in the frameworks of the two languages. While simple variables, equations and logical statements may be easy to convert, most advanced functionality comes from the various classes available in the .NET and JRE frameworks. 

Because these two frameworks are dramatically different and extremely dynamic, creating a tool to convert from one to the other would be a very difficult undertaking. 

Also included in the difficulty of writing a conversion tool is deciphering the various complex casting assignments, and translating the pass by reference capabilities in C\# to the purely pass by value functionality in Java. Because of the limitations to conversion software in general, and the perceived need to carefully review all translated code, conversion tools were not used on this project. 

In addition to the technical limitations of conversion software, valuable learning experiences which help in the overall understanding of how the software operates would be removed by using this software. Spending time in the code itself has allowed for learning the code well enough to examine various options for enhancement.



\section{Basic Structure}\label{sec:basicStructure}
At the core of the Band Diagram program there are three object types called \texttt{Metal}, \texttt{Dielectric} and \texttt{Semiconductor}. These store the attributes of the material they represent from the available materials, as well as the \emph{thickness} attribute. In the case of the \texttt{Dielectric}, various optional charge values for locations internal to the material are stored. These objects also have a storage area for result data. 

All three of these objects are then can be referred to through the parent class named \texttt{Material} which contains identifying information about the type of object stored within, and contains a normalized interface to the three types of objects for use in the charting. After the individual \texttt{Metal}, \texttt{Dielectric} and \texttt{Semiconductor} objects are created, they are stored as a list of \texttt{Material} objects that represent the structure. 

Once the structure is complete, the list is passed to the \texttt{Kernel} object to be evaluated using the options selected by the user. As the values are calculated, they are stored inside the individual objects contained in the \texttt{Material} containers. Once the calculations are complete, these values can be accessed and displayed in chart or report form. There are various other objects in the project, but these core objects are used to represent and calculate the data. These objects have been maintained from the C\# version and carried over to the Java version largely unchanged. While the frameworks for displaying forms and other user interface controls are different between the languages, the structure of these core objects could be maintained without much change. This provided the opportunity to limit testing in these core calculation areas to only syntax and minor logic bugs without the necessity to redesign the algorithms used.



\section{Chart Improvements}\label{sec:chartImprovements}
The original Band Diagram software uses a custom charting solution. This solution was able to display the basic charts, with multiple colors, using a resizable canvas. Shapes were available for displaying the banding in the dielectrics while mouse pointer position information was available to give more exact measurements. 

These charts had all the needed functionality and allowed for maximum flexibility in creating the desired displays. However, said functionality came at the expense of eliminating extra features and the bug hunting support of a third-party solution. 

Because Java was designed from the ground up to be a cross-platform language, most third-party libraries were also designed to be cross-platform, or at least have a cross-platform version available for use. Using one of these solutions allowed for the potential of increased stability and added useful features with no additional development time. 

Realistically, using a third-party solution could result in a greatly reduced cost due to shorter implementation time and the ability to pass discovered bugs back to the original development team for that product. The JFreeChart project was selected due to the open source licensing and the large feature set and charting options. By utilizing this library we were able to add support for the following: 

\begin{enumerate}
\item Different series colors, 
\item Zooming,
\item Panning, and
\item More traditional hover tips.
\end{enumerate}

without spending the time to implement these features directly. Side benefits include a very well structured code base that allows for easy expansion, adding different chart types, or customizing existing charts as needed. 

By reducing the time necessary to implement the basic functionality of the charts, more time was made available to improve the functionality in other areas.

In practice the JFreeCharts library was an excellent choice for the static charts and provided a very quick and feature-rich view into the data. The relative time spent implementing the chart section of the new Band software compared with the rest of the project was very small, and assuredly smaller than the time to implement the basic charting functionality in the original Band software. While there are problems that can occur when using third-party tools, like bugs or design limitations, the JFreeChart libraries worked very well for the static charts. 

That being said, the JFreeCharts package was not designed to be used for animated charts that require speed. In fact, the JFreeCharts development toolkit documentation specifically states that results will not be good if a frame rate of more than one chart per minute is desired! This limitation was due to the package being designed originally with only static charts in mind. While the event layers were later added to accommodate detecting and displaying changes in the dataset, the graphical portion of the code updates the chart drawing by repainting the entire chart image as if it was just created. 

While this possibility was alarming, the estimation proved to be a conservative view with the implementation in the new Band software. Decent refresh rates were achieved by breaking the dataset updating code out into a separate thread. Separate threads would typically run on a separate processor from the main application as most desktop systems have multiple cores and/or processors today.



\section{Standardize User Interface}\label{sec:standardizeUserInterface}
A few elements of the original Band Diagram program were not created in accordance with current graphical user interface standards. This re-write provided a good opportunity to fix these issues, making the software easier and more efficient to use.

\begin{figure}[ht]
\begin{center}
\includegraphics*[width=7.0in,height=5.0in,keepaspectratio]{movieParam.eps}
\end{center}
\caption{Original Band Diagram Movie Parameters Form}
\label{fig:movieParam}
\end{figure}

The first change was to move the output section of the Movie Parameters dialog (Figure~\ref{fig:movieParam}) to a new Export Tools dialog. This allows the Output Parameters subsection to be more easily found by newer users so they can benefit from this functionality. The movie controls were then changed so the Movie Parameters settings only work with the animation functionality on the main page, and the settings in the Export Tools dialog only relate to the export data generation. This change will also be more intuitive for existing users.

Allowing the user to stop the chart animation was a second change that could potentially add benefit. The Band Diagram software allows the user to select the range and number of steps in the animation. If the user selects a large number of steps in the animation, or if the structure is complex, this can cause the movie to require a substantial amount of time to generate, rendering the software unresponsive until the process is finished. By adding a stop button to the main form while the animation was running, said upgrade provided a more user-friendly feel to the application. That change was also facilitated by the multi-threading change discussed in the last section. It allowed the user interface events to continue to react quickly to user actions while chart animation updated as fast as possible in the background. 

Other changes were related to the unofficial standard operation that has become expected behavior from professional applications today such as:
\begin{enumerate}
\item The option to double-click and edit any item displayed in a list view,
\item Using default \emph{Enter} and \emph{Escape} keys for the \emph{OK} and \emph{Cancel} buttons,
\item Changing frequently used fields to be the default starting point for editing the form.
\end{enumerate}

These changes were added to provide the consistency users have come to expect.

To facilitate these changes, NetBeans form builder tools were used to design and develop the interfaces for the new system. The NetBeans form tools are based on the freely available Matisse GUI Builder for Java. These tools provide a visual way to display and manipulate a Swing-based Java GUI without the need to hand-code each piece. In addition to the visual benefits of this tool, it also provides valuable integration with the Java Bean Binding technology which allows objects to be directly connected to and updated from an interface. 

A simple example of this technology can be found in the \emph{Add/Edit Metal} form in the new Band program. Before this form is displayed, a \texttt{Metal} object is instantiated with default values and attached to the form using Bean Binding. With this technique, each field on the form can be linked with a variable in the \texttt{Metal} object by using the standard getter and setter methods available in that class. Each time the form is updated the object is also updated. Finally when the form closes, it can determine whether the newly created object is valid or invalid. If the object is valid, it can be added to or it can update the user-created structure or the master Metals list by inserting or replacing the old object with the newly created one. This eliminates the need for the class calling the form to handle the data coming from the form directly.



\section{Faster execution}\label{sec:fasterExecution}
In general, the C\# version of Band is fast. However, it can slow down if there are a large number of materials in the structure. Another source of slow down and long wait times is the exploration of newer features, such as multiple semiconductor materials in one structure.

In order to combat these speed issues the data processing algorithms have been broken out into a separate thread in the Java version of Band.

Future work in this area may be necessary as the software continues to grow with new, more complex features. It may be necessary to split the processing into multiple threads to allow the user to benefit more fully from a multi-processor system. It also may be valuable to explore GPU computing options such as nVidia's CUDA framework to make use of the video card. Such use could further increase processing speed.



\section{Feature Review}\label{sec:featureReview}
Most of the functionality presented in the C\# Band version has been represented in the Java version. The charting features have been greatly enhanced with the Java version due to the use of the JFreeChart library. Added features include lasso zoom, scroll wheel zoom, panning, and added material colors for the main charts. More options are available in the right click menu for each chart to allow the user to directly print or save the image as JPG or PNG format images. Options are also available to modify the title, axis descriptions and chart background colors directly from the right-click menu. Most importantly, the software will now run natively on any operating system where the JRE is available. Microsoft Windows, Linux and Mac OS are supported.

A new user data files feature has also been added. When a user runs the application for the first time a \emph{.Band3} directory is created in their user account. Copies of the material files are then copied into this directory for use while running the software. 

If the user adds, edits or removes any material data the users files will be modified. This allows multiple people to use the software on the same machine without interfering with each others data. If a user wants to transfer their customized material files to another machine they simply need to copy this directory from their user account on one machine to another. This copy will also work from one OS to another, as long as the files are placed in the proper user directory for that OS. If at any time the user wants to revert back to the original data files, said user can simply remove this directory. The files will be recreated the next time the program executes.



\section{Ease of installation and upgrade}\label{sec:easeOfInstallation}
The installation software chosen for this project was the Web Start Technology compatible with Java version 1.2.1 or later. The Web Start Technology comes preloaded with all Java JRE installations 1.6 and higher. Because it was expected this project would be installed cross-platform, a cross-platform installer was necessary from the start. 

Many cross-platform installers require the JRE to be pre-installed as they, themselves, are written in Java. Other solutions would require separate compiles for each operating system and/or a potentially high cost to use the installation software. 

By using the built-in Web Start Technology we assume that Java 1.6 or the Web Start runtime is installed on the users' system. By providing instructions to download the JRE on the distribution website, users can complete this initial task themselves. Once a JRE that is version 1.6 or higher is installed, then the user can download and install the Band software through Web Start by clicking on a single link. 

NetBeans provides an automatic build process for the Web Start installer and a sample web page to use to install the link on the download page. When the user installs a Java Web Start application, all of the same installer functionality exists as in more traditional installers. A \emph{Start} menu folder and Desktop shortcut can be created.

When the user runs the software from the shortcut, the background Web Start Technology layer detects if the user is on the Internet. If so, after querying the user for permission, it will attempt to check and download any newer version of the application or supporting libraries that are found on the download server. If the user is not connected to the Internet, the software will execute locally and check for updates at a later time. Using Java Web Start avoided the need to maintain separate update functionality as in the previous version of Band. 

If the user wishes to uninstall the software, they can go to the standard area for doing so in their operating system, such as \emph{Add/Remove Programs} in Microsoft Windows. The user could then remove the software as in a standard installation. The software cache would be removed and the user can then reinstall at a later time, if desired.



\section{Releases}\label{sec:releases}

In undertaking the conversion of the Band diagram software, two releases, the Direct Port and the Enhanced Release, were made available.

\subsection{Direct Port (3.0)}\label{sub:directPort}
The first Java release was developed to resemble the C\# release as much as possible. Only minor interface changes were made including double-click functionality in list views and default key functionality. Some controls, such as the toolbar, appear dissimilar when comparing C\# and Java. Using C\#, the toolbar automatically removes items as the window resizes, and creates a button to access the removed items in a drop-down window. Java does not have this functionality in their native controls, so the fields will remain in their constant locations in the toolbar. If the form is resized too small, they will be hidden. Because a large screen size is not needed to support all fields, this change should not create an issue. If problems are found with this implementation, custom Java controls can be explored to duplicate the missing functionality.

Owing to the fact that the goal was to have the switch to Java be the only change the user needed to learn for this version, the first release was designed to allow users to feel comfortable in the new version. Everything was placed where users would remember from the previous version, and functionality was matched.


\subsection{Enhanced Release (3.1)}
An enhanced version will be released that will make more dramatic changes to the user interface and feature set. The release will incorporate the larger interface changes, such as renaming the \emph{Movie/Output Parameters} section. Added features from branched C\# versions will be added to this version, and the new forms and options will be made available. 

This could include the dielectric band rounding and the multiple semiconductor support currently being developed. The multiple semiconductor support will require additional multi-threading support and options to allow the user to manually weigh the speed versus accuracy trade-off. Other new interface functionality could also be explored, such as drag and drop \texttt{Materials}, additional chart types and possible reporting outputs.


\subsection{Code Obfuscation}\label{sub:codeObfuscation}
It is important to note that while the software will be released via GPL 2.0, many of the experimental, newer features have not yet been released to the public. If there is a desire to release these features before a paper can be published on them, code obfuscation methods can be explored to allow a release of the Java binary files that cannot be reverse-engineered easily. This will allow users to increase input into the new features while protecting the interest of the researchers involved. Once the new features have been properly explored in published works, the features will be released into the general GPL 2.0 code base and the code obfuscation can be removed.



\section{Documentation and Maintenance}\label{sec:docsMaintain}
All code documentation will use the popular documentation tool called javadoc, the comments can be compiled into an easier-to-read HTML format. That can be viewed as a document set when reviewing the software. 

These comments are also used by the IDE such as NetBeans or Eclipse, to provide in-line documentation while using the code from the project. This makes future reuse and expansion much easier. Because this project will likely get passed on to future students, this element is important to limit the time new programmers will need to become familiar with the code base. As the software changes and evolves, these new changes can be made available to easy use and review through future javadoc comments.




%%%%%%%%%%%%%%%%%%%%%%%%%%%%%%%%%%%%%%%%%%%%%%%%%%%%%%%%%%%%%%%%%%%%%%%%%%%%%%
%
% Chapter: Observations from the Conversion Process
%
%%%%%%%%%%%%%%%%%%%%%%%%%%%%%%%%%%%%%%%%%%%%%%%%%%%%%%%%%%%%%%%%%%%%%%%%%%%%%%
\chapter{Observations from the Conversion Process}\label{ch:observations}

This section discusses the "lessons learned" from this project. While some areas are specific to the Band software, many could be helpful to other conversion projects, specifically those related to Java and C\#.



\section{Coding Comparison}\label{sec:codingComparison}
Java and C\# are very similar languages. It is thought by many that C\# was Microsoft's answer to Java, and incorporated many style choices made in the Java language by Sun Microsystems.

One of the most notable differences in the languages, however, is the support for pointers found in C\#. Java, in passing everything by value, has no real method of passing a reference. Because of this difference, code changes were made to certain interfaces, especially in the \texttt{Kernel} class. Those changes allowed the functions to pass back the proper object as needed, instead of filling a reference. 

While some view this difference as a limitation in Java, the simplicity allows for increased readability in the Java code base.

Another difference between the two languages is found in string manipulation methods. C\# has more base string manipulation methods found in their \texttt{String} class than Java. 

In the case of the Band program, the methods \texttt{indexOfAny} and \texttt{lastIndexOfAny} were used in the \texttt{Kernel} class. The \texttt{indexOfAny} method tests a set of character values with a \texttt{String} to see if any of the characters can be found in the string. If one is found, the \texttt{String} index of the first one found is returned to the caller.

The \texttt{lastIndexOfAny} works similarly to the \texttt{indexOfAny} function. However, it returns the \texttt{String} index of the last character found. This function, while valuable in the case of the Band program, does not seem to be common functionality and was likely left out of the Java \texttt{String} class for the purpose of simplification. In the Java version of the Band program, equivalent functions were created and placed in the \texttt{Kernel} class to accomplish the same task.

Most of the forms in the original Band software were coded by hand, meaning each field was considered a text field. Any entries were converted to their appropriate type, added to an object of the right type for the form, and then added to either the structure list or the master list of that type of material.

The Band program consists of many lists of materials created from the \texttt{List} objects stored. In the original Band program, for each object in the list of materials, each value was pulled from the object and added to the proper cell to generate the list view. Then, when a user clicks a row that index is used to retrieve the proper object for replacement or manipulation. 

In Java, many of these functions have been anticipated and methods have been provided to make them simpler. We have already discussed Bean Binding in relation to individual fields and objects, but Bean Binding also comes in handy when displaying a list of objects. By attaching the list view to the list of objects, and then assigning each column in the list view to a certain field, you can easily display data for all the objects in a list. 

Options are available to limit the number of fields that are displayed for each object, additional options are available that relate to how the lists could possibly be manipulated to update the objects in the list. When a row in the list is selected, instead of getting an index to the original list, the object that row represents is returned for manipulation. 

Because the original data structures in Java are not event-aware, wrappers have been created for use with Java Beans. These wrappers are called Observable Objects. The Observable Objects fire events whenever an object in them changes, or if the order or number of objects in the list changes. If one of these lists is used in conjunction with the list view, the list view will automatically notice an object has been updated and requery the appropriate objects to update the list view. This method allows for a much more intuitive, direct relationship between the object and the list view displaying it. 

This same technique can be layered.  One example of layering can be seen in the case with the \texttt{Dielectric} objects in the Band software. 

First, it is possible to select a \texttt{Dielectric} object from the list and enter into the editing screen for that object. Internal to the object is a list of \texttt{EvalPoint} objects storing the locations for various injected charges. By using this same list view population method again in the dielectric editing form, we can display the point list in a similar list view as well. 

It is important to note that while this method can be quite useful, the list that is directly linked to the list view must exist at all times the list view is instantiated, even if empty. For this reason, when manipulating the \texttt{Dielectric} objects, the object must be passed to the form before the form is made visible.



\section{JFreeChart Animation}\label{sec:jfreechartAnimation}
The animation capabilities of the JFreeChart package were also utilized in this project. However, speed improvements may be necessary to approach a comparable frame rate to the C\# version. Changes in the JFreeChart code may be necessary to enforce redrawing only affected parts of the chart, instead of the method currently used by JFreeCharts which consists of repainting the entire chart for each dataset change. Different methods for this approach are already being explored as patches by other users in the JFreeChart forums and bug tracking systems.

The frame rates are not as smooth as the original application, there is still room for improvement with the implementation of the JFreeCharts libraries. Three obvious possibilities are: 

\begin{enumerate}
\item Delaying the chart refresh detection until the dataset has completed updating the dataset to prevent delays from the update detection code.
\item A version of JFreeCharts has been submitted to the bug tracking and patch section of the web page that has yet to be reviewed. That version will only refresh the changed portions of the chart when the dataset changes. If this code is accepted into the main trunk of the project and the new Band software recompiled with the new library a very noticeable improvement in speed may be seen. There wasn't time for deep review of the changes made available in the patch.
\item Changes could also be implemented directly before they are reviewed by the JFreeCharts team if increased speed is deemed necessary for the new version before the patch version is available in production.
\end{enumerate}

\section{Object Serialization}\label{sec:objectSerialization}

One of the main frustrations while working with the Java Band software was the design decision to maintain the serialization of objects for saving state. In the original C\# program, and the current Java version, materials are stored into lists and then serialized to disk for later retrieval and use. 

During the early stages of the software development cycle, the serialization was a very easy way to save information without the requirement to add a lot of code for this purpose. As the code entered the debugging phase, however, the limitations of this method were made apparent. Subtle changes to the material objects destroyed the objects ability to read in old format objects from disk, and populate them back into new format objects. 

It was possible to work around these issues and maintain the serializing version of the code, however, it would be much more flexible to convert the objects to a standard text format, such as XML, where the object data can be parsed and placed into a new object, or written out independently of the actual memory format of the object. This would both allow for easier future changes, and minor changes to be made to the object structures by hand since the XML code would be human readable. It would also allow other programs the opportunity to view and write out data in the Band format for future use in the Band software without needing to understand the object structures.

\section{Java Frameworks}\label{sec:javaFrameworks}

The use of the Swing framework and, in turn, the JFreeCharts charting package were great successes in this project. Both frameworks are relatively easy to use and are very well structured. It is abundantly apparent that much forethought went into the design of both frameworks. These frameworks provided a good foundation, and in the case of the Swing framework, there was no limit of help available online to explain more advanced implementations.




%%%%%%%%%%%%%%%%%%%%%%%%%%%%%%%%%%%%%%%%%%%%%%%%%%%%%%%%%%%%%%%%%%%%%%%%%%%%%%
%
% Chapter: Conclusions
%
%%%%%%%%%%%%%%%%%%%%%%%%%%%%%%%%%%%%%%%%%%%%%%%%%%%%%%%%%%%%%%%%%%%%%%%%%%%%%%
\chapter{Conclusions}\label{ch:conclusions}

This project has resulted in a Java version of the Band Diagram software that can be used on Mac OS and Linux without error while still maintaining the Microsoft Windows support. The project has added additional charting functionality and better user file support and removed the need for internal maintenance for the charting and update functionalities. By using the Java language, this software can easily be passed on to other students for maintenance as Java is one of the most popular languages in use today in the software engineering industry.



\section{Future Directions}\label{sec:futureDirections}

Conversion to Java has created broad opportunity for added features. Previous sections have discussed potential future features such as XML-based configuration and GPU processing to enhance the application. Along with these features, there are various other additions that could benefit the users.

One exciting update would be additional chart types such as those found in Figure~\ref{fig:samples}. Using the JFreeCharts as a base, it would be easy to support displaying multiple charts at the same time. A multi-chart view, with all four main charts, would be an ideal feature to allow the user to see everything associated with the structure at once. It would also lend itself to changing the structure in a separate or side window while viewing the charts, thereby creating more of an experimentation dashboard. Said dashboard could display the various effects of changes more quickly than opening and editing through dialogs. 

Additional 3D charts, which are currently in an experimental stage in JFreeCharts, could allow graphing 3 different variables against each other. This may lead to further discovery through previously unseen correlations between attributes.

\begin{figure}[ht]
% Code in figures is normally not centered, but graphical figures are
% (the \centering command works as well as the 'center' environment, 
% because its scope is limited to the figure).
\begin{center}
\includegraphics*[width=6.0in,height=4.0in,keepaspectratio]{fig3a.eps}
\end{center}
\begin{center}
\includegraphics*[width=6.0in,height=4.0in,keepaspectratio]{fig3b.eps}
\end{center}
\caption{Examples of additional JFreeChart chart types}
\label{fig:samples}
\end{figure}


Added interface improvements could make the software even easier to use. An example would be to allow drag and drop functionality while creating a structure. If the user could:

\begin{enumerate}
\item Drag a new material to the structure and automatically get a window to set the material details, 
\item If the user could drag materials up and down the structure list,
\end{enumerate}

it would allow for increased speed while creating structures.

Another beneficial feature could be an option to save the chart animations to a video format for emailing or future reference outside of the software. An integrated help system or tutorial could also be useful as users learn how the program works. 

Finally, it may be beneficial to recompile the code in an applet format to allow users to create structures from the web and export their data and images without ever needing to install the software. This would help users who are not allowed to download and install the software themselves, due to network or some other restrictions. It would allow for client-side processing of structure band data in these situations.



\backmatter

%%%%%%%%%%%%%%%%%%%%%%%%%%%%%%%%%%%%%%%%%%%%%%%%%%%%%%%%%%%%%%%%%%%%%%%%%%%%%%
%
% Bibilography
%
%%%%%%%%%%%%%%%%%%%%%%%%%%%%%%%%%%%%%%%%%%%%%%%%%%%%%%%%%%%%%%%%%%%%%%%%%%%%%%
\bibliographystyle{plain}

\begin{thebibliography}{99} 

\label{references}

\bibitem{band:web1} Knowlton Research Group. {\em Band Diagram Program.}\\
\texttt{http://nano.boisestate.edu/bandDiagramProgram}

\bibitem{band:web2} Mono Project. \texttt{http://www.mono-project.com}

\bibitem{band:paper1} Richard G. Southwick III, and William B. Knowlton, Stacked Dual Oxide MOS Energy Band Diagram Visual Representation Program, Invited Paper, IEEE Transactions on Device and Materials Reliability, 6(2) (2006) p. 136-145.

\bibitem{band:paper2} R. G. Southwick III, A. Sup, A. Jain and W. B. Knowlton. An Interactive Simulation Tool for Complex Multilayer Dielectric Devices. {\em Manuscript submitted to IEEE Transactions on Electron Devices}

\bibitem{band:web3} JFreeChart. \texttt{http://www.jfree.org/jfreechart}

\bibitem{band:web4} The Comprehensive Perl Archive Network. \texttt{http://cpan.perl.org}

\bibitem{band:web5} Computer Language Benchmarks Game. \texttt{http://shootout.alioth.debian.org}

\end{thebibliography}

\finish

\end{document}
